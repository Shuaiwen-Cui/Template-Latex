%% 
%% Copyright 2019-2021 Elsevier Ltd
%% 
%% This file is part of the 'CAS Bundle'.
%% --------------------------------------
%% 
%% It may be distributed under the conditions of the LaTeX Project Public
%% License, either version 1.2 of this license or (at your option) any
%% later version.  The latest version of this license is in
%%    http://www.latex-project.org/lppl.txt
%% and version 1.2 or later is part of all distributions of LaTeX
%% version 1999/12/01 or later.
%% 
%% The list of all files belonging to the 'CAS Bundle' is
%% given in the file `manifest.txt'.
%% 
%% Template article for cas-sc documentclass for 
%% single column output.

\documentclass[a4paper,fleqn]{cas-sc}

% If the frontmatter runs over more than one page
% use the longmktitle option.

%\documentclass[a4paper,fleqn,longmktitle]{cas-sc}

% \usepackage[numbers]{natbib}
\usepackage[authoryear]{natbib}
%\usepackage[authoryear,longnamesfirst]{natbib}

%%%Author macros
\def\tsc#1{\csdef{#1}{\textsc{\lowercase{#1}}\xspace}}
\tsc{WGM}
\tsc{QE}
%%%

% Uncomment and use as if needed
%\newtheorem{theorem}{Theorem}
%\newtheorem{lemma}[theorem]{Lemma}
%\newdefinition{rmk}{Remark}
%\newproof{pf}{Proof}
%\newproof{pot}{Proof of Theorem \ref{thm}}

\begin{document}
\let\WriteBookmarks\relax
\def\floatpagepagefraction{1}
\def\textpagefraction{.001}
%\let\printorcid\relax % 可去掉页面下方的ORCID(s)

% Short title
% \shorttitle{<short title of the paper for running head>}    
\shorttitle{Leveraging social media news to predict stock index movement using RNN-boost}   

% Short author
% \shortauthors{<short author list for running head>} 
\shortauthors{V. {{\=A}}nand Rawat et al.}

% Main title of the paper
\title[mode = title]{Leveraging social media news to predict stock index movement using RNN-boost}  

% Title footnote mark
% eg: \tnotemark[1]
% \tnotemark[<tnote number>] 
\tnotemark[1,2]

% Title footnote 1.
% eg: \tnotetext[1]{Title footnote text}
% \tnotetext[<tnote number>]{<tnote text>} 
\tnotetext[1]{This document is the results of the research project funded by the National Science Foundation.}
\tnotetext[2]{The second title footnote which is a longer text matter to fill through the whole text width and overflow into another line in the footnotes area of the first page.}

% First author
%
% Options: Use if required
% eg: \author[1,3]{Author Name}[type=editor,
%       style=chinese,
%       auid=000,
%       bioid=1,
%       prefix=Sir,
%       orcid=0000-0000-0000-0000,
%       facebook=<facebook id>,
%       twitter=<twitter id>,
%       linkedin=<linkedin id>,
%       gplus=<gplus id>]

% \author[<aff no>]{<author name>}[<options>]

% Corresponding author indication
% \cormark[<corr mark no>]

% Footnote of the first author
% \fnmark[<footnote mark no>]

% Email id of the first author
% \ead{<email address>}

% URL of the first author
% \ead[url]{<URL>}

% Credit authorship
% eg: \credit{Conceptualization of this study, Methodology, Software}
% \credit{<Credit authorship details>}

% Address/affiliation
% \affiliation[<aff no>]{organization={},
%             addressline={}, 
%             city={},
% %          citysep={}, % Uncomment if no comma needed between city and postcode
%             postcode={}, 
%             state={},
%             country={}}

% \author[<aff no>]{<author name>}[<options>]

% Footnote of the second author
% \fnmark[2]

% Email id of the second author
% \ead{}

% URL of the second author
% \ead[url]{}

% Credit authorship
% \credit{}

% Address/affiliation
% \affiliation[<aff no>]{organization={},
%             addressline={}, 
%             city={},
% %          citysep={}, % Uncomment if no comma needed between city and postcode
%             postcode={}, 
%             state={},
%             country={}}

% Corresponding author text
% \cortext[1]{Corresponding author}

% Footnote text
% \fntext[1]{}

% For a title note without a number/mark
%\nonumnote{}

\author[1,3]{V. {{\=A}}nand Rawat}[type=editor,
    auid=000,bioid=1,
    prefix=Sir, 
    role=Researcher, 
    orcid=0000-0001-7511-2910]
\cormark[1] 
\fnmark[1] 
\ead{cvr_1@tug.org.in} 
\ead[url]{www.cvr.cc,www.tug.org.in}
\credit{Conceptualization of this study, Methodology, Software}

\author[2,4]{Han Theh Thanh}[style=chinese]

\author[2,3]{T. Rishi Nair}[role=Co-ordinator, suffix=Jr]
\fnmark[2] 
\ead{rishi@sayahna.org}
\ead[URL]{www.sayahna.org}
\credit{Data curation, Writing - Original draft preparation}

\author[1,3]{Karl Berry}
\cormark[2] 
\fnmark[1,3]
\ead{karl@freefriends.org} 
\ead[URL]{www.tug.org}

\address[1]{Indian \TeX{} Users Group, Trivandrum 695014, India}
\address[2]{Sayahna Foundation, Jagathy, Trivandrum 695014, India}
\address[3]{\TeX{} Users Group, Providence, MA, USA}

\cortext[1]{Corresponding author} 
\cortext[2]{Principal corresponding author} 

% Here goes the abstract
\begin{abstract}
In this work we demonstrate $a_b$ the formation Y\_1 of a new type of polariton on the interface between a cuprous oxide slab and a polystyrene micro-sphere placed on the slab. The evanescent field of the resonant whispering gallery mode of the micro sphere has a substantial gradient, and therefore effectively couples with the quadrupole $1^S$ excitons in cuprous oxide. This evanescent polariton has a long life-time, which is determined only by its excitonic and component. The polariton lower branch has a well pronounced minimum. This suggests that this excitation is localized and can be utilized for possible. The spatial coherence of the polariton can be improved by assembling the micro-spheres into a linear chain.
\end{abstract}

% Use if graphical abstract is present
%\begin{graphicalabstract}
%\includegraphics{}
%\end{graphicalabstract}

% Research highlights
\begin{highlights}
\item highlight-1
\item highlight-2
\item highlight-3
\end{highlights}

% Keywords
% Each keyword is seperated by \sep
\begin{keywords}
keyword-1 \sep 
keyword-2 \sep 
keyword-3
\end{keywords}

\maketitle

% Main text
\section{Section-1}

Text of section-1 \cite{Fortunato2010}.

\section{Section-2}

Text of section-2 \cite{NewmanGirvan2004}.

\section{Section-3}

Text of section-3 \cite{Vehlowetal2013}.

% Numbered list
% Use the style of numbering in square brackets.
% If nothing is used, default style will be taken.
%\begin{enumerate}[a)]
%\item 
%\item 
%\item 
%\end{enumerate}  

% Unnumbered list
%\begin{itemize}
%\item 
%\item 
%\item 
%\end{itemize}  

% Description list
%\begin{description}
%\item[]
%\item[] 
%\item[] 
%\end{description}  

% Figure
% \begin{figure}[<options>]
% 	\centering
% 		\includegraphics[<options>]{}
% 	  \caption{}\label{fig1}
% \end{figure}


% \begin{table}[<options>]
% \caption{}\label{tbl1}
% \begin{tabular*}{\tblwidth}{@{}LL@{}}
% \toprule
%   &  \\ % Table header row
% \midrule
%  & \\
%  & \\
%  & \\
%  & \\
% \bottomrule
% \end{tabular*}
% \end{table}

% Uncomment and use as the case may be
%\begin{theorem} 
%\end{theorem}

% Uncomment and use as the case may be
%\begin{lemma} 
%\end{lemma}

%% The Appendices part is started with the command \appendix;
%% appendix sections are then done as normal sections
%% \appendix

% To print the credit authorship contribution details
% \printcredits

%% Loading bibliography style file
%\bibliographystyle{model1-num-names}
\bibliographystyle{cas-model2-names}

% Loading bibliography database
\bibliography{cas-refs}

% Biography
% \bio{}
% Here goes the biography details.
% \endbio

% \bio{pic1}
% Here goes the biography details.
% \endbio

\end{document}

